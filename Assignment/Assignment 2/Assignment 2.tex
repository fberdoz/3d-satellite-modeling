\documentclass[11pt, oneside]{article}  	% use "amsart" instead of "article" for AMSLaTeX format
\usepackage{geometry}                		% See geometry.pdf to learn the layout options. There are lots.
\geometry{a4paper}                   		% ... or a4paper or a5paper or ... 
%\geometry{landscape}                		% Activate for rotated page geometry
\usepackage[parfill]{parskip}    			% Activate to begin paragraphs with an empty line rather than an indent
\usepackage{graphicx}				% Use pdf, png, jpg, or eps� with pdflatex; use eps in DVI mode
								% TeX will automatically convert eps --> pdf in pdflatex		


\graphicspath{ {images/} }

\usepackage{fancyhdr}
\pagestyle{fancy}
\usepackage[latin1]{inputenc} 

% Prof. Forbes math packages
\usepackage{amsmath} % cmex10
\usepackage{amssymb}
\usepackage{amsthm}
\usepackage{bm}
\usepackage{mathrsfs}
\usepackage{wrapfig}

% Matrix command
\newcommand{\bma}[1]{\left[\begin{array}{#1}}
\newcommand{\ema}{\end{array}\right]}
\newcommand{\trans}{{\ensuremath{\mathsf{T}}}} % transpose
\newcommand{\utimes}{ {\raisebox{-0.6ex}{ \kern-1.0ex\raisebox{0.6ex}{ \small $\mathsf{v}$}}} } % 
\newcommand{\onehalf}{\mbox{$\textstyle{\frac{1}{2}}$}}


% Bold symbols
\DeclareMathAlphabet{\mbf}{OT1}{ptm}{b}{n} % for bold face Roman
\newcommand{\mbs}[1]{{\boldsymbol{#1}}} % for bold face Greek

% Other bold symbols 
\newcommand{\mbfbar}[1]{{\bar{\mbf{#1}}}}
\newcommand{\mbfhat}[1]{{\hat{\mbf{#1}}}}
\newcommand{\mbftilde}[1]{{\tilde{\mbf{#1}}}}
\newcommand{\mbsbar}[1]{{\bar{\boldsymbol{#1}}}}
\newcommand{\mbshat}[1]{{\hat{\boldsymbol{#1}}}}
\newcommand{\mbstilde}[1]{{\tilde{\boldsymbol{#1}}}}

% Physical Space, physical vectors, a vectrix, etc. 
\newcommand{\pspace}{\mathbb{P}} 
\newcommand{\ura}[1]{{\underrightarrow{{#1}}}}
\newcommand{\vectrix}[1]{\ensuremath \underrightarrow{\boldsymbol{\mathcal{F}}}_{#1}}
\def\fdota{{\raisebox{-2pt}{\LARGE $\cdot$}}}
\def\fdotb{{\raisebox{-0.6ex}{ \kern0.2ex\raisebox{0.8ex}{\tiny $\hspace*{-1ex}\circ$}}}}
\def\fddota{{\raisebox{-2pt}{\LARGE $\cdot\hspace*{-0.2ex}\cdot$}}}
\def\fddotb{{\raisebox{-0.6ex}{ \kern0.2ex\raisebox{0.8ex}{\tiny $\hspace*{-1ex}\circ\circ$}}}}
\newcommand{\fdot}[1]{{^{\fdota{\mbox{\footnotesize${#1}$}}}}}
\newcommand{\fddot}[1]{{^{\fddota{\mbox{\footnotesize${#1}$}}}}}


% Short form for equations
\newcommand{\beq}{\begin{equation}}
\newcommand{\eeq}{\end{equation}}
\newcommand{\bdis}{\begin{displaymath}}
\newcommand{\edis}{\end{displaymath}}
\newcommand{\beqarray}{\begin{eqnarray}}
\newcommand{\eeqarray}{\end{eqnarray}}
\newcommand{\beqarraynn}{\begin{eqnarray*}}
\newcommand{\eeqarraynn}{\end{eqnarray*}}

%Must be equal to ...
\newcommand{\mbeq}{\overset{!}{=}}

% Cross operator
\newcommand{\crossop}[3]{\bma{ccc} 0 & -#3 & #2 \\ #3 & 0 & -#1 \\ -#2 & #1 & 0 \ema}
\newcommand{\matr}[9]{\bma{ccc} #1 & #2 & #3 \\ #4 & #5 & #6 \\ #7 & #8 & #9 \ema}
\newcommand{\colvec}[3]{\bma{c} #1 \\ #2 \\ #3 \ema}
\newcommand{\rowvec}[3]{\bma{ccc} #1 & #2 & #3 \ema}

\lhead{\footnotesize MECH 642\\Advanced Dynamics}
\rhead{\footnotesize Assignment 2\\ Fr�d�ric Berdoz, 260867318}

\begin{document}

\title{Assignment 2}
\author{Fr�d�ric Berdoz\\260867318}
\date{}

\maketitle

% Question 1 ----------------------------------------------------------------------------------------------------------------------------------------------------------
\section{}
From the definition of $\ura{u}$ and $\ura{v}$, we have:
$$\mbf{u}_a=
\bma{c}
	4\\
	u_{a2}\\
	-2
\ema, \qquad
\mbf{v}_a=
\bma{c}
	1\\
	3\\
	5
\ema
$$
\paragraph{a)}
Using the definition of the dot product:
$$
\ura{u}\cdot\ura{v}=\mbf{u}_a^\trans\mbf{v}_a=4+3u_{a2}-10=-6+3u_{a2}\mbeq12
$$
Solving for $u_{a2}$:
$$u_{a2}=\frac{12+6}{3}=6$$

\paragraph{b)}
Using the definition of the cross product:
$$\ura{u}\times\ura{v}=\vectrix{a}^\trans\mbf{u}_a^\times\mbf{v}_a=\vectrix{a}^\trans\bma{ccc} 0 & 2 & u_{a2} \\ -2 & 0 & -4 \\ -u_{a2}& 4 & 0 \ema\bma{c}1\\3\\5\ema=\vectrix{a}^\trans\bma{c}6+5u_{a2}\\ -22 \\ -u_{a2}+12\ema\mbeq\vectrix{a}^\trans\bma{c}16 \\ -22 \\ 10 \ema$$
Solving for $u_{a2}$, we have the following two equations:
$$u_{a2}=\frac{16-6}{5}=2$$
$$u_{a2}=-(10-12)=2$$
Therefore, $u_{a2}=2$ is the unique solution.

% Question 2 ----------------------------------------------------------------------------------------------------------------------------------------------------------
\section{}
\begin{align*}
\mbox{det }\mbs{\omega}_a^\times & =
\mbox{det}\crossop{\omega_{a1}}{\omega_{a2}}{\omega_{a3}}=
\bma{c}
	0\\
	\omega_{a3}\\
	-\omega_{a2}
\ema^\trans
\bma{c}
	-\omega_{a3}\\
	0\\
	\omega_{a1}
\ema^\times
\bma{c}
	\omega_{a2}\\
	-\omega_{a1}\\
	0
\ema
\\ & =
\bma{ccc}
	0 & \omega_{a3} & -\omega_{a2}
\ema
\matr{0}{-\omega_{a1}}{0}{\omega_{a1}}{0}{\omega_{a3}}{0}{-\omega_{a3}}{0}
\bma{c}
	\omega_{a2}\\
	-\omega_{a1}\\
	0
\ema
\\ & =
\bma{ccc}
	0 & \omega_{a3} & -\omega_{a2}
\ema
\bma{c}
	\omega_{a1}^2 \\
	\omega_{a1}\omega_{a2} \\
	\omega_{a3}\omega_{a1}
\ema
=\omega_{a3}\omega_{a1}\omega_{a2}-\omega_{a2}\omega_{a3}\omega_{a1}=0
\end{align*}
Therefore, $\mbs{\omega}_a^\times$ is not invertible.

% Question 3 ----------------------------------------------------------------------------------------------------------------------------------------------------------
\section{}
\paragraph{a)}
By inspection, we can see that:
$$r_{b1}=r_{a1}=1, \quad r_{b2}=r_{a3}=4, \quad r_{b3}=-r_{a2}=-3$$
Therefore, $\mbf{r}_b=\bma{ccc} 1 & 4 & -3 \ema^\trans$

\paragraph{b)}
$\mathcal{F}_b$ is obtained by rotating $\mathcal{F}_a$ about $\ura{a}^1$ by $90�$ .

\begin{align*}
\mbf{C}_{ba} & = \vectrix{b}\cdot \vectrix{a}^\trans=\colvec{\ura{b}^1}{\ura{b}^2}{\ura{b}^3}\cdot\rowvec{\ura{a}^1}{\ura{a}^2}{\ura{a}^3}
=\matr{\ura{b}^1\cdot\ura{a}^1}{\ura{b}^1\cdot\ura{a}^2}{\ura{b}^1\cdot\ura{a}^3}{\ura{b}^2\cdot\ura{a}^1}{\ura{b}^2\cdot\ura{a}^2}{\ura{b}^2\cdot\ura{a}^3}{\ura{b}^3\cdot\ura{a}^1}{\ura{b}^3\cdot\ura{a}^2}{\ura{b}^3\cdot\ura{a}^3} 
\\ & =
\matr{1}{0}{0}{0}{0}{1}{0}{-1}{0}
\end{align*}
where the dot products where found by looking at Figure 1.
\paragraph{c)}
$$\mbf{r}_b=\mbf{C}_{ba}\mbf{r}_a=\matr{1}{0}{0}{0}{0}{1}{0}{-1}{0}\colvec{1}{3}{4}=\colvec{1}{4}{-3}$$
This corresponds to the solution found in a).

% Question 4 ----------------------------------------------------------------------------------------------------------------------------------------------------------
\section{} 
\begin{align*}
\mbox{det }\mbf{Q} & =\mbox{det}\matr{0}{\onehalf}{-\frac{\sqrt{3}}{2}}{1}{0}{0}{0}{\frac{\sqrt{3}}{2}}{\onehalf}
=\colvec{0}{1}{0}^\trans\colvec{\onehalf}{0}{\frac{\sqrt{3}}{2}}^\times\colvec{-\frac{\sqrt{3}}{2}}{0}{\onehalf}
\\ & =
\rowvec{0}{1}{0}\matr{0}{-\frac{\sqrt{3}}{2}}{0}{\frac{\sqrt{3}}{2}}{0}{-\onehalf}{0}{\onehalf}{0}\colvec{-\frac{\sqrt{3}}{2}}{0}{\onehalf}
\\ & =
\rowvec{0}{1}{0}\colvec{0}{-1}{0}=-1
\end{align*}
In conclusion, $\mbf{Q}$ is not a valid direction cosine matrix because its determinant is different than  +1.

% Question 5 ----------------------------------------------------------------------------------------------------------------------------------------------------------
\section{}
From the definition of the cross product:
$$
\ura{u}\times\ura{v}=\vectrix{a}^\trans\mbf{u}_a^\times\mbf{v}_a=\vectrix{b}^\trans\mbf{u}_b^\times\mbf{v}_b
$$
Multiplying on the left by $\vectrix{b}$ (dot product), we get:
\beq
\vectrix{b}\cdot\vectrix{a}^\trans\mbf{u}_a^\times\mbf{v}_a=\vectrix{b}\cdot\vectrix{b}^\trans\mbf{u}_b^\times\mbf{v}_b
\label{eq:1}
\eeq
Recall:
\beq
\vectrix{b}\cdot\vectrix{a}^\trans=\mbf{C}_{ba}
\label{eq:2}
\eeq
\beq
\vectrix{b}\cdot\vectrix{b}^\trans=\mbf{1}
\label{eq:3}
\eeq
Substituting \eqref{eq:2} and \eqref{eq:3} into \eqref{eq:1}, we obtain:
\beq
\mbf{C}_{ba}\mbf{u}_a^\times\mbf{v}_a=\mbf{u}_b^\times\mbf{v}_b
\label{eq:4}
\eeq
Now, from the definition of the DCM:
\beq
\mbf{v}_a=\mbf{C}_{ab}\mbf{v}_b=\mbf{C}_{ba}^\trans\mbf{v}_b
\label{eq:5}
\eeq
\beq
\mbf{u}_b=\mbf{C}_{ba}\mbf{u}_a
\label{eq:6}
\eeq
Substituting \eqref{eq:5} and \eqref{eq:6} into \eqref{eq:4}, we obtain:
\beq
\mbf{C}_{ba}\mbf{u}_a^\times\mbf{C}_{ba}^\trans\mbf{v}_b=(\mbf{C}_{ba}\mbf{u}_a)^\times\mbf{v}_b
\label{eq:7}
\eeq
Since \eqref{eq:7} must be valid for any $\ura{v}$ (hence any $\mbf{v}_b$): $$\mbf{C}_{ba}\mbf{u}_a^\times\mbf{C}_{ba}^\trans=(\mbf{C}_{ba}\mbf{u}_a)^\times \quad \Box$$

% Question 6 ----------------------------------------------------------------------------------------------------------------------------------------------------------
\section{}
\paragraph{a)}
We have:
\beq
	\ura{T}=\vectrix{a}^\trans\mbf{T}_a\vectrix{a}=\vectrix{b}^\trans\mbf{T}_b\vectrix{b}
	\label{eq:a}
\eeq
Taking the dot product on the left with $\vectrix{b}$ and on the right with $\vectrix{b}^\trans$:
\bdis
\underbrace{\vectrix{b}\cdot\vectrix{a}^\trans}_{\mbf{C}_{ba}}\mbf{T}_a\underbrace{\vectrix{a}\cdot\vectrix{b}^\trans}_{\mbf{C}_{ab}}=\underbrace{\vectrix{b}\cdot\vectrix{b}^\trans}_{\mbf{1}}\mbf{T}_b\underbrace{\vectrix{b}\cdot\vectrix{b}^\trans}_{\mbf{1}}
\edis
Where we used the definition of the direction cosine matrix. Rewriting both sides:
\bdis
\mbf{C}_{ba}\mbf{T}_a\mbf{C}_{ab}=\mbf{T}_b \quad \Box
\edis
\paragraph{b)}
Using \eqref{eq:a} and $\ura{u}=\vectrix{a}^\trans\mbf{u}_a$, we find:
\bdis
\ura{T}\cdot\ura{u}=\vectrix{b}^\trans\mbf{T}_b\underbrace{\vectrix{b}\cdot\vectrix{a}^\trans}_{\mbf{C}_{ba}}\mbf{u}_a=\vectrix{b}^\trans\mbf{T}_b\mbf{C}_{ba}\mbf{u}_a \quad \Box
\edis

% Question 7 ----------------------------------------------------------------------------------------------------------------------------------------------------------
\section{}
\paragraph{a)}
\begin{align*}
(\lambda-1)(\lambda^2+\lambda(1-\mbox{tr}\mbf{C}_{qp})+1) & =
(\lambda-1)\lambda^2+(\lambda-1)(\lambda-\lambda\mbox{tr}\mbf{C}_{qp})+(\lambda-1) 
\\ & =
\lambda^3-\lambda^2+\lambda^2-\lambda^2\mbox{tr}\mbf{C}_{qp}-\lambda+\lambda\mbox{tr}\mbf{C}_{qp}+\lambda-1
\\ & =
\lambda^3-\lambda^2\mbox{tr}\mbf{C}_{qp}+\lambda\mbox{tr}\mbf{C}_{qp}-1
\\ & =
0 \quad \Box
\end{align*}
One obvious solution of this equation is $\lambda_1=+1$, which is therefore also an eigenvalue of $\mbf{C}_{qp}$.
\paragraph{b)}
Recall that the trace is a linear operator, i.e. $\forall \alpha, \beta \in \mathbb{C}, \forall \mbf{A}, \mbf{B} \in \mathbb{C}^{n\times n}, \, n\in \mathbb{N}$: $$\mbox{tr}(\alpha\mbf{A}+\beta\mbf{B})=\alpha\mbox{tr}\mbf{A}+\beta\mbox{tr}\mbf{B}$$
Therefore, we can express the trace of $\mbf{C}_{qp}$ as follows:
\beq
\mbox{tr}\mbf{C}_{qp} = \cos\phi\,\mbox{tr}(\mbf{1})+(1-\cos\phi)\mbox{tr}(\mbf{aa}^\trans)-\sin\phi\,\mbox{tr}(\mbf{a}^\times)
\label{eq:b}
\eeq
We can now compute the trace of each element.

\begin{itemize}
\item Trivially:
\beq
\mbox{tr}(\mbf{1})=3
\label{eq:c}
\eeq
\item Let $\mbf{a}=\rowvec{a_1}{a_2}{a_3}^\trans$, therefore:
$$
\mbf{aa}^\trans=\colvec{a_1}{a_2}{a_3}\rowvec{a_1}{a_2}{a_3}
=\matr{a_1^2}{a_1a_2}{a_1a_3}{a_2a_1}{a_2^2}{a_2a_3}{a_3a_1}{a_3a_2}{a_3^2}
$$
By inspection: 
\beq
\mbox{tr}(\mbf{aa}^\trans)=a_1^2+a_2^2+a_3^2={\Vert\ura{a}\Vert}_2=1
\label{eq:d}
\eeq
\item Taking the same generic notation for $\mbf{a}$:
$$
\mbf{a}^\times=\crossop{a_1}{a_2}{a_3}
$$
By inspection: 
\beq
\mbox{tr}(\mbf{a}^\times)=0
\label{eq:e}
\eeq
\end{itemize}
Substituting \eqref{eq:c}, \eqref{eq:d} and \eqref{eq:e} into \eqref{eq:b}:
\bdis
\mbox{tr}\mbf{C}_{qp} = 3\cos\phi+(1-\cos\phi)=1+2\cos\phi \quad \Box
\edis
Using this result and the fact that $\mbf{C}_{qp}\in\mbox{SO(3)}$, we have the two following conditions:
\beq
\mbox{det}\mbf{C}_{qp}=\lambda_1\lambda_2\lambda_3=1
\label{eq:f}
\eeq
\beq
\mbox{tr}\mbf{C}_{qp}=\lambda_1+\lambda_2+\lambda_3=1+2\cos\phi
\label{eq:g}
\eeq
Since we have already found that $\lambda_1=1$, \eqref{eq:f} and  \eqref{eq:g} become: 
\beq
\lambda_2\lambda_3=1
\label{eq:h}
\eeq
\beq
\lambda_2+\lambda_3=2\cos\phi
\label{eq:i}
\eeq
From \eqref{eq:i}, we find:
\beq
\lambda_3=2\cos\phi-\lambda_2
\label{eq:j}
\eeq
Substituting \eqref{eq:j} into \eqref{eq:h}:
\beq
\lambda_2(2\cos\phi-\lambda_2)=1
\label{eq:k}
\eeq
We can now rearrange \eqref{eq:k} as follows:
\beq
\lambda_2^2-2\cos\phi\lambda_2+1=0
\label{eq:l}
\eeq
This is simply a quadratic equation that we can solve as follows:
\begin{align*}
\lambda_2 & =
\cos\phi\pm\sqrt{\cos^2\phi-1}=\cos\phi\pm\sqrt{(-1)(1-\cos^2\phi)}
\\ & =
\cos\phi\pm \sqrt{-1}\sqrt{\sin^2\phi}=\cos\phi\pm i\sin\phi=e^{\pm i\phi}
\label{eq:m}
\end{align*}
Substituting this solution into \eqref{eq:h}:
$$e^{\pm i\phi}\lambda_3=1\quad\Rightarrow\quad\lambda_3=\frac{1}{e^{\pm i\phi}}=e^{\mp i\phi}$$
In other words, $\lambda_2$ is the complex conjugate of $\lambda_3$ and vice versa. It doesn't matter which one has the positive sign in the exponent.
% Question 8 ----------------------------------------------------------------------------------------------------------------------------------------------------------
\section{}
First, let's derive some useful properties:
$$
\mbf{a}^\times\mbf{a}^\times=\mbf{aa}^\trans-\mbf{a}^\trans\mbf{a1}=\mbf{aa}^\trans-\mbf{1}
$$
$$
(\mbf{aa}^\trans-\mbf{1})^2=\mbf{a}\overbrace{\mbf{a}^\trans\mbf{a}}^{=\ 1}\mbf{a}^\trans-2\mbf{aa}^\trans+\mbf{1}=\mbf{1}-\mbf{a}\mbf{a}^\trans
$$
$$
(\mbf{aa}^\trans-\mbf{1})^3=(\mbf{aa}^\trans-\mbf{1})^2(\mbf{aa}^\trans-\mbf{1})=(\mbf{1}-\mbf{a}\mbf{a}^\trans)(\mbf{aa}^\trans-\mbf{1})=\mbf{aa}^\trans-\mbf{1}
$$
$$
\vdots$$
$$
(\mbf{aa}^\trans-\mbf{1})^n=(-1)^n(\mbf{1}-\mbf{aa}^\trans), \quad n\in\mathbb{N}\setminus\{0\}
$$
Moreover:
$$
\mbf{a}^\times(\mbf{1}-\mbf{aa}^\trans)=\mbf{a}^\times-\overbrace{\mbf{a}^\times\mbf{a}}^{=\,\mbf{0}}\mbf{a}^\trans=\mbf{a}^\times
$$
Now, let's expand the expression for $e^{-\phi\mbf{a}^\times}$ using the definition of the matrix exponential and the former properties:
\begin{align*}
e^{-\phi\mbf{a}^\times} 
& =
\sum_{k=0}^{\infty}\frac{(-\phi)^k}{k!}(\mbf{a}^\times)^k
\\ & = 
\sum_{l=0}^{\infty}\frac{\overbrace{(-1)^{2l}}^{=\ 1}\phi^{2l}}{(2l)!}(\mbf{a}^\times)^{2l}
+\sum_{m=0}^{\infty}\frac{\overbrace{(-1)^{2m+1}}^{=-1}\phi^{2m+1}}{(2m+1)!}(\mbf{a}^\times)^{2m+1}
\\ & =
\sum_{l=0}^{\infty}\frac{\phi^{2l}}{(2l)!}(\overbrace{\mbf{a}^\times\mbf{a}^\times}^{=\ \mbf{aa}^\trans-\mbf{1}})^{l}
-\sum_{m=0}^{\infty}\frac{\phi^{2m+1}}{(2m+1)!}\mbf{a}^\times(\overbrace{\mbf{a}^\times\mbf{a}^\times}^{=\ \mbf{aa}^\trans-\mbf{1}})^{m}
\\ & =
\mbf{1}+\sum_{l=1}^{\infty}\frac{\phi^{2l}}{(2l)!}\overbrace{(\mbf{aa}^\trans-\mbf{1})^{l}}^{=\,(-1)^l(\mbf{1}-\mbf{aa}^\trans)}
-\left(\phi\mbf{a}^\times+\sum_{m=1}^{\infty}\frac{\phi^{2m+1}}{(2m+1)!}\mbf{a}^\times\overbrace{(\mbf{aa}^\trans-\mbf{1})^{m}}^{=\,(-1)^m(\mbf{1}-\mbf{aa}^\trans)}\right)
\\ & =
\mbf{1}+\overbrace{\sum_{l=1}^{\infty}\frac{(-1)^l\phi^{2l}}{(2l)!}}^{=\ \cos\phi\,-1}(\mbf{1}-\mbf{aa}^\trans)
-\left(\phi\mbf{a}^\times+\sum_{m=1}^{\infty}\frac{(-1)^m\phi^{2m+1}}{(2m+1)!}\overbrace{\mbf{a}^\times(\mbf{1}-\mbf{aa}^\trans)}^{=\,\mbf{a}^\times}\right)
\\ & =
\mbf{1}+(\cos\phi-1)(\mbf{1}-\mbf{aa}^\trans)
-\overbrace{\sum_{m=0}^{\infty}\frac{(-1)^m\phi^{2m+1}}{(2m+1)!}}^{=\,\sin\phi}\mbf{a}^\times
\\ & =
\cos\phi\mbf{1}+(1-\cos\phi)\mbf{aa}^\trans-\sin\phi\mbf{a}^\times \quad \Box
\end{align*}
\end{document}


